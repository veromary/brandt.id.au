\input opustex
\input opusgreg
\input eplain

\font\bigtype = hmin10 scaled \magstep2
\font\Bigtype = hmin17 scaled \magstep2

\font\musicaps = Priory scaled \magstep5
\font\textcaps = Priory scaled \magstep4

\musicinitialfont\musicaps
\textinitialfont\textcaps

\let\augmentum=\gobble
\let\augmentumduplex=\gobbletwo
\let\episem=\gobbletwo

\special{papersize=148mm,210mm}

\hoffset1.5mm
\voffset-7mm
\hsize100mm
\vsize173mm

\abovesystem1
\belowsystem8

\font\itlyrics = cmti9
\font\rmlyrics = cmr9
\font\smallrm = cmr8
\font\smold = cmmi10 scaled 900

\def\source#1\par{\rightline{\smallrm #1}}


\def\engl#1{\cchar{-7}{\itlyrics #1}}

\def\beginhymn #1.{\centerline{\bigtype {#1}}\par\bigskip}

\def\mycr{\par\vskip-\parskip}

\def\beginlyrics{\begingroup
                       \def\\{\mycr}
                       \pretolerance=10000
                       \raggedright
                       \parskip=5pt
                       \parindent-3mm}
\def\endlyrics{\endgroup}

\def\verse#1. {\hskip2mm\llap{{\oldstyle #1}. }}

\nopagenumbers


\newbox\linkbox
\newbox\wordbox
%\newdimen\boxwid

\def\lyrlink{%
  % Bogen erstellen: Put the smilie in the box:
  \setbox\linkbox=\hbox{$\smile$}%
  % In Box der Breite eines Wortzwischenraums einsetzen:
  % Make the box one space wide:
  \setbox\linkbox=\hbox to\the\fontdimen2\the\font{%
    \hss
    % Unter die Grundlinie druecken:
    % Lower the box under the word:
    \lower\ht\linkbox\hbox{%
      % Zusaetzlicher vertikaler Abstand zur Wortunterseite:
      % lower the smile a little bit more:
      \lower1pt\hbox{%
	\relax
	  \hbox{$\smile$}%
	}}%
    \hss}%
  % Keine zusaetzliche Tiefe fuer Bogen anrechnen:
  % Make the depth of the box zero:
  \dp\linkbox=0pt
  % Bogen setzen:
  \box\linkbox}

\def\intralink#1{%
  \setbox\linkbox=\hbox{$\smile$}%
  \setbox\wordbox=\hbox{#1}%
  \setbox\linkbox=\hbox to\wd\wordbox{%
    \hss
    % Unter die Grundlinie druecken:
    % Lower the box under the word:
    \lower\ht\linkbox\hbox{%
      % Zusaetzlicher vertikaler Abstand zur Wortunterseite:
      % lower the smile a little bit more:
      \lower1pt\hbox{%
	\relax
	  \hbox{$\smile$}%
	}}%
    \hss}%
  % Keine zusaetzliche Tiefe fuer Bogen anrechnen:
  % Make the depth of the box zero:
  \dp\linkbox=0pt
%  \let\boxwid=\the\wd\wordbox
%  \lower\ht\linkbox%
  \box\wordbox \llap{\box\linkbox}}


\beginhymn Stabat Mater dolorosa.


\setgregorian1
\setclef14
\musicindent14mm
\raisesong3\Internote
\initiumgregorianum
\musicinitial{VI}{S}%
\sgn {T}{a}{b}\punctum M\egn
\sgn {}at\punctum N\egn
\spatium
\sgn M{a}t\engl{\idx{At the Cross} her station keeping}\punctum a\egn
\sgn {}er\punctum N\egn
\spatium
\sgn dol\punctum a\egn
\sgn {}o-\punctum c\egn
\sgn r{\'o}s\bmolle b\punctum b\egn
\sgn {}a{}\punctum a\augmentum b\egn
\spatium
\divisiominor
\spatium
\sgn Jux\punctum a\egn
\sgn ta{}\punctum N\egn
\spatium
\sgn {Cr}{u}c\punctum M\egn
\sgn {}em\punctum L\egn
\spatium
\sgn lac\engl{Stood the mournful mother weeping,}\punctum K\egn
\sgn rim\punctum L\egn
\sgn {}{\'o}s\punctum K\egn
\sgn {}a,\punctum J\augmentum J\egn
\spatium
\divisiominor
\custos N
\lineaproxima
\sgn Dum\punctum N\egn
\spatium
\sgn pen\punctum M\egn
\sgn d{\'e}b\engl{Close to Jesus to the last.}\punctum N\egn
\sgn {}at\punctum a\egn
\spatium
\sgn F{\'\i}-\punctum N\egn
\sgn li-\punctum M\egn
\sgn {}u{s.}\punctum M\augmentum N\egn
\spatium
\Finisgregoriana

\bigskip

\beginlyrics

\hbox{

\parskip=3pt

\vtop{
\hsize50mm

\rmlyrics %%for the latin

%\verse1. St\'abat M\'ater dolor\'osa\\
%Juxta cr\'ucem lacrim\'osa,\\
%Dum pend\'ebat F\'\i lius.

\verse2. Cujus \'animam gem\'entem,\\
Contrist\'atem et dol\'entem\\
Pertrans\'\i vit gl\'adius.

\verse3. O quam tristis et affl\'\i cta\\
Fuit illa bened\'\i cta\\
Mater Unig\'eniti!

\verse4. Qu\ae\ m\ae r\'ebat et dol\'ebat,\\
Pia Mater, dum vid\'ebat\\
Nati p\oe nas \'\i nclyti.

\verse5. Quis est homo qui non fleret,\\
Matrem Christi si vid\'eret\\
In tanto suppl\'\i cio?

\verse6. Quis non posset contrist\'ari,\\
Christi Matrem contempl\'ari\\
Dol\'entem cum F\'\i lio?

\verse7. Pro pecc\'atis su\ae\ gentis,\\
Vidit Jesum in torm\'entis,\\
Et flag\'ellus s\'ubditum.

\verse8. Vidit suum dulcem Natum\\
Mori\'endo desol\'atum,\\
Dum em\'\i sit sp\'\i ritum.

\verse9. Eia Mater, fons am\'oris,\\
Me sent\'\i re vim dol\'oris\\
Fac, ut tecum l\'ugeam.

}

\vtop{
\hsize50mm

\itlyrics

%\verse1. \idx{By the Cross her vigil keeping},\\
%Stood the mournful mother weeping,\\
%Close to Jesus to the last.

\verse2. Through her heart, His sorrow sharing,\\
All His bitter anguish bearing,\\
Now at length the sword had pass'd.

\verse3. Oh, how sad and sore distress\'ed\\
Was that mother highly bless\'ed\\
Of the sole-begotten One!

\verse4. Christ above in torment hangs;\\
She beneath beholds the pangs\\
Of her dying glorious Son.

\verse5. Is there one who would not weep,\\
Whelm'd in miseries so deep,\\
Christ's dear Mother to behold?

\verse6. Can the human heart refrain\\
From partaking in her pain\\
In that Mother's pain untold?

\verse7. Bruis'd, derided, curs'd, defil'd,\\
She beheld her tender Child,\\
All with bloody scourges rent.

\verse8. For the sins of His own nation,\\
Saw Him hang in desolation,\\
Till His spirit forth He sent.

\verse9. O thou Mother! fount of love!\\
Touch my spirit from above,\\
Make my heart with thine accord.

}

}

\medskip

\hbox{

\parskip3pt

\vtop{
\hsize50mm

\rmlyrics

\verse10. Fac ut \'ardeat cor meum\\
In am\'ando Christum Deum,\\
Ut sibi compl\'aceam.

\verse11. Sancta Mater, istud agas,\\
Crucif\'\i xi fige plagas\\
Cordi meo v\'alide.

\verse12. Tui Nati vulner\'ati,\\
Tam dign\'ati pro me pati,\\
P\oe nas mecum d\'\i vide.

\verse13. Fac me tecum pie flere,\\
Crucif\'\i xo condol\'ere,\\
Donec ego v\'\i xero.

\verse14. Juxta Crucem tecum stare,\\
Et me tibi soci\'are\\
In planctu des\'\i dero.

\verse15. Virgo v\'\i rginem pr\ae cl\'ara,\\
Mihi jam non sis am\'ara:\\
Fac me tecum pl\'angere.

\verse16. Fac ut portem Christi mortem\\
Passi\'onis fac cons\'ortem,\\
Et plagas rec\'olere.

\verse17. Fac me plagis vulner\'ari,\\
Fac me Cruce\lyrlink inebri\'ari,\\
Et cru\'ore F\'\i lii.

\verse18. Flammis ne urar succ\'ensus,\\
Per te, Virgo, sim def\'ensus\\
In die jud\'\i cii.

\verse19. Christe, cum sit hinc ex\'\i re,\\
Da per Matrem me ven\'\i re\\
Ad palmam vict\'ori\ae .

\verse20. Quando corpus mori\'etur,\\
Fac ut \'anim\ae\ don\'etur\\
Parad\'\i si gl\'oria.  Amen.
}

\hskip5mm
\vtop{
\hsize50mm

\itlyrics

\verse10. Make me feel as thou hast felt:\\
Make my soul to glow and melt\\
With the love of Christ my Lord.

\verse11. Holy Mother, pierce me through,\\
In my heart each wound renew\\
Of my Saviour crucified.

\verse12. Let me share with thee His pain,\\
Who for all my sins was slain,\\
Who for me in torments died.

\verse13. Let me mingle tears with thee.\\
Mourning him who mourned for me,\\
All the days that I may live.

\verse14. By the Cross with thee to stay,\\
There with thee to weep and pray,\\
All the days that I may live.

\verse15. Virgin of all virgins blest,\\
Listen to my fond request:\\
Let me share thy grief divine.

\verse16. Let me, to my latest breath,\\
In my body bear the death\\
Of that dying Son of thine.

\verse17. Wounded with His every wound,\\
Steep my soul till it has swoon'd\\
In His very Blood away.

\verse18. Be to me, O Virgin, nigh\\
Lest in  flames I burn and die,\\
In His awful Judgement day.

\verse19. Christ, when Thou shalt call me hence,\\
Be Thy Mother my defence,\\
Be Thy cross my victory.

\verse20. While my body here decays,\\
May my soul Thy goodness praise,\\
Safe in Paradise with Thee. Amen.


}
}

\medskip

\source Ascribed to Jacapone da Todi, {\smold13}th century

\source Translation Fr.~E.~Caswall {\smold 1814}--{\smold 1878}

\endlyrics

\end